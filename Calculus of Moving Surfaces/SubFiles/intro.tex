\documentclass[../main.tex]{subfiles}


\begin{document}

    Recent attempts to the solution of the eigenvalue problem \cite{grinfeld2004laplacian} \cite{grinfeld2012laplace} of the Laplacian in regular polygons have partially shifted their attention to the calculus of moving surfaces. The calculus of moving surfaces undoubtedly offers a geometric insight to the problem which can be exploited in a straightforward fashion to aide to the solution of the problem. Through this technique a few illuminating results have been proven. It is possible to prove, and a proof will in fact be offered in this paper, that by re scaling the polygons such that their area is the same as the one of the circle, the first term to appear in the $\frac{1}{N}$ expansion of the eigenvalue is that of power three. Moreover it is possible to substantiate analytically the values for the expansion, and to definitively assert their connection to the Riemann zeta function. The calculus of moving surfaces is thus a powerful tool that may be applied to many other boundary perturbation and optimization problems. In the next few sections a breif account of the theory will be given, after which it will be applied to the problem in question.

\end{document}